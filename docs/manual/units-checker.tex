\htmlhr
\chapter{Units Checker\label{units-checker}}

For many applications, it is important to use the correct units of measurement
for primitive types. For example, NASA's Mars Climate Orbiter (cost: \$327
million) was lost because of a discrepancy between use of the metric unit
Newtons and the imperial measure Pound-force.

The \emph{Units Checker} ensures consistent usage of units. For example,
consider the following code:

\begin{alltt}
@m int meters = 5 * UnitsTools.m;
@s int secs = 2 * UnitsTools.s;
@mPERs int speed = meters / secs;
\end{alltt}

Due to the annotations \<@m> and \<@s>, the variables \code{meters} and
\code{secs} are guaranteed to contain only values with meters and seconds as
units of measurement. Utility class \code{UnitsTools} provides constants with
which the dimensionless integers are multiplied to get values of the
corresponding unit. The assignment of a dimensionless value to \code{meters}, as
in \code{meters = 99}, will be flagged as an error by the Units Checker.

The division \code{meters/secs} takes the types of the two operands into account
and determines that the result is of type meters per second, signified by the
\code{@mPERs} qualifier. We provide an extensible framework to define the result
of operations on units.


\section{Units annotations\label{units-annotations}}

The checker currently supports four varieties of annotations:

\begin{enumerate}

\item dimension annotations
(\refqualclass{checker/units/qual}{Length},
\refqualclass{checker/units/qual}{Mass}, \dots)

\item unit annotations
(\refqualclass{checker/units/qual}{m},
\refqualclass{checker/units/qual}{kg}, \dots)

\item polymorphic annotations
(\refqualclass{checker/units/qual}{PolyUnit})

\item special annotations
(\refqualclass{checker/units/qual}{UnknownUnits},
\refqualclass{checker/units/qual}{UnitsBottom},
\refqualclass{checker/units/qual}{Dimensionless})

\end{enumerate}

\subsection{Dimension annotations\label{unit-dimension-annotations}}

Dimension annotations can be used to declare what the expected unit of
measurement is, without fixing the particular unit used.

For example, one could write a method taking a \code{@Length} value, without
specifying whether it will take meters or kilometers. The following dimension
annotations are provided by the Units Checker:

\begin{description}
\item[\refqualclass{checker/units/qual}{Acceleration}]

\item[\refqualclass{checker/units/qual}{Angle}]

\item[\refqualclass{checker/units/qual}{Area}]

\item[\refqualclass{checker/units/qual}{Current}]

\item[\refqualclass{checker/units/qual}{Length}]

\item[\refqualclass{checker/units/qual}{Luminance}]

\item[\refqualclass{checker/units/qual}{Mass}]

\item[\refqualclass{checker/units/qual}{Speed}]

\item[\refqualclass{checker/units/qual}{Substance}]

\item[\refqualclass{checker/units/qual}{Temperature}]

\item[\refqualclass{checker/units/qual/time/duration}{TimeDuration}]

\item[\refqualclass{checker/units/qual/time/instant}{TimeInstant}]

\item[\refqualclass{checker/units/qual}{Volume}]
\end{description}

Note that TimeDuration represents the elapsed time between two events, where as
TimeInstant represents the precise time of the occurence of some event. This
conceptual separation is intentially designed into the Units Checker to support
the Java 8 Time API.

Users can define and use additional dimension annotations with the Units
Checker.

\subsection{Unit annotations\label{unit-annotations}}

For each dimension, Units Checker provides a subset of the corresponding SI
units:

For \code{@Acceleration}:
\begin{enumerate}
\item Meter per Second squared \refqualclass{checker/units/qual}{mPERs2}
\end{enumerate}

For \code{@Angle}:
\begin{enumerate}
\item Radians \refqualclass{checker/units/qual}{radians},
\item Degrees \refqualclass{checker/units/qual}{degrees}
\end{enumerate}

For \code{@Area}:
\begin{enumerate}
\item Square Meters \refqualclass{checker/units/qual}{m2},
\item Square Millimeters \refqualclass{checker/units/qual}{mm2},
\item Square Kilometers \refqualclass{checker/units/qual}{km2}
\end{enumerate}

For \code{@Current}:
\begin{enumerate}
\item \textbf{Ampere} \refqualclass{checker/units/qual}{A}
\end{enumerate}

For \code{@Length}:
\begin{enumerate}
\item \textbf{Meters} \refqualclass{checker/units/qual}{m},
\item Millimeters \refqualclass{checker/units/qual}{mm},
\item Kilometers \refqualclass{checker/units/qual}{km},
\end{enumerate}

For \code{@Luminance}:
\begin{enumerate}
\item \textbf{Candela} \refqualclass{checker/units/qual}{cd},
\end{enumerate}

For \code{@Mass}:
\begin{enumerate}
\item \textbf{Grams} \refqualclass{checker/units/qual}{g},
\item Kilograms \refqualclass{checker/units/qual}{kg}
\end{enumerate}

For \code{@Speed}:
\begin{enumerate}
\item Meter per Second \refqualclass{checker/units/qual}{mPERs} and
\item Kilometer per Hour \refqualclass{checker/units/qual}{kmPERh}
\end{enumerate}

For \code{@Substance}:
\begin{enumerate}
\item \textbf{Mole} \refqualclass{checker/units/qual}{mol}
\end{enumerate}

For \code{@Temperature}:
\begin{enumerate}
\item \textbf{Kelvin} \refqualclass{checker/units/qual}{K}
\item Celsius \refqualclass{checker/units/qual}{C}
\end{enumerate}

For \code{@TimeDuration}:
\begin{enumerate}
\item \textbf{Seconds} \refqualclass{checker/units/qual/time/duration}{s},
\item Milliseconds \refqualclass{checker/units/qual/time/duration}{ms},
\item Nanoseconds \refqualclass{checker/units/qual/time/duration}{ns},
\item Minutes \refqualclass{checker/units/qual/time/duration}{min},
\item Hours \refqualclass{checker/units/qual/time/duration}{h},
\item \dots
\end{enumerate}

See package \refpackage{checker/units/qual/time/duration}
{org.checkerframework.checker.units.qual.time.duration} for the rest of the
units.

For \code{@TimeInstant}:
\begin{enumerate}
\item Seconds \refqualclass{checker/units/qual/time/instant}{CALs},
\item Milliseconds \refqualclass{checker/units/qual/time/instant}{CALms},
\item Nanoseconds \refqualclass{checker/units/qual/time/instant}{CALns},
\item Minutes \refqualclass{checker/units/qual/time/instant}{CALmin},
\item Hours \refqualclass{checker/units/qual/time/instant}{CALh},
\item \dots
\end{enumerate}

See package \refpackage{checker/units/qual/time/instant}
{org.checkerframework.checker.units.qual.time.instant} for the rest of the
units.

Note that TimeInstant units follow a naming convention of prefixing their
respective TimeDuration units with \code{CAL}.

For \code{@Volume}:
\begin{enumerate}
\item Cubic Meters \refqualclass{checker/units/qual}{m3},
\item Cubic Millimeters \refqualclass{checker/units/qual}{mm3},
\item Cubic Kilometers \refqualclass{checker/units/qual}{km3}
\end{enumerate}

For the SI base units (\code{@s}, \code{@m}, \code{@g}, \code{@A}, \code{@K},
\code{@mol}, \code{@cd}; indicated in bold above), you may specify a SI unit
prefix using enumeration \code{\refclass{checker/units/qual}{Prefix}}. These
annotations take an optional \code{Prefix} enum as argument.

For example, to use nanoseconds as a unit, you can use \code{@s(Prefix.nano)} as
a unit type.

Some of the annotations are aliases to their prefixed base unit annotations. For
example, \<@mm> is an alias of \<@m(Prefix.milli)> and the Units Checker regards
the use of either of these two annotations as identical to each other.

Like dimension annotations, users can define and use additional custom unit
annotations with the Units Checker.

\subsection{Polymorphic annotations\label{unit-poly-annotations}}

The polymorphic annotation \refqualclass{checker/units/qual}{PolyUnit} enables
you to write a method that takes an argument of any unit type and returns a
result of that same type. For more about polymorphic qualifiers, see
Section~\ref{qualifier-polymorphism}. For an example of its use, see the
\href{../api/org/checkerframework/checker/units/qual/PolyUnit.html}{\<@PolyUnit>
Javadoc}.

\subsection{Special annotations\label{unit-special-annotations}}

The annotation \refqualclass{checker/units/qual/}{UnknownUnits} is the top type
of the type hierarchy for the Units Checker. A code element with this annotation
has a unit which is not represented by any of the annotations in use by the
Units Checker.

For example, the standard Units Checker provides the speed annotation
\<@kmPERh>, but does not provide the acceleration annotation \<@kmPERh2> since
it is not commonly used as an acceleration unit. If the user write the following
code:

\begin{alltt}
@kmPERh carSpeed = 50 * UnitsTools.kmPERh;
carAcceleration = carSpeed / (2 * UnitsTools.h);
\end{alltt}

The resulting unit of dividing \<@carSpeed> by 2 hours is \<@UnknownUnits> since
kilometer per hour squared is not represented by any annotations.

\<@UnknownUnits> is the default qualifier for local variables, exceptions, and
implicit upper bounds <T> for checked code. It is the default qualifier for
implicit and explicit upper bounds <T extends Object> for byte code.

The annotation \refqualclass{checker/units/qual/}{UnitsBottom} is the bottom
type of the type hierarchy for the Units Checker. The null literal and the Void
class always has this type. It is rarely necessary to use this annotation in
code.

The annotation \refqualclass{checker/units/qual/}{Dimensionless} represents
dimensionless quantities. It is the default qualifier for all other code
elements, such as number and string literals, fields, method parameters and
returns, \dots.

\section{UnitsTools Utility Class\label{units-tools}}

Class \code{UnitsTools} contains a constant for each dimension and unit. To
create a value of the particular unit, multiply a dimensionless value with one
of these constants, as shown at the beginning of this chapter.

By using static imports, this allows very natural notation; for example, after
statically importing \code{UnitsTools.m}, the expression \code{5 * m} represents
five meters.

As all these unit constants are public, static, and final with value one, the
compiler will optimize away these multiplications.

\section{Extending the Units Checker\label{extending-units}}

You can create new dimension annotations and unit annotations that are specific
to the particular needs of your project. An easy way to do this is by copying
and adapting an existing annotation.

Here is an example of a new unit annotation for Hertz:

\begin{alltt}
@Documented
@Retention(RetentionPolicy.RUNTIME)
@Target(\ttlcb{}ElementType.TYPE_USE, ElementType.TYPE_PARAMETER\ttrcb{})
@SubtypeOf(\ttlcb{}TimeDuration.class\ttrcb{})
@UnitsMultiple(quantity = s.class, prefix = Prefix.pico)
public @interface ps \ttlcb{}\ttrcb{}
\end{alltt}

The \code{@SubtypeOf} meta-annotation specifies that this annotation introduces
an additional unit of time duration.

The \code{@UnitsMultiple} meta-annotation specifies that this annotation should
be a pico multiple of the basic unit \code{@s}: \code{@ps} and
\code{@s(Prefix.pico)} behave equivalently and interchangeably. Most annotation
definitions do not have a \<@UnitsMultiple> meta-annotation.

Note that all custom annotations must have the
\<@Target(\ttlcb{}ElementType.TYPE\_USE, ElementType.TYPE\_PARAMETER\ttrcb{})>
meta-annotation. See section \ref{creating-define-type-qualifiers}.

If the unit has a multiplicative or divisive relationship with other units, it
must be stated in a \code{@UnitsRelations} meta-annotation. For example, for the
meters unit, we state the following relationships:

\begin{alltt}
@UnitsRelations(\ttlcb{}
  @Relation(op = Op.MUL, lhs = m.class, rhs = m.class, res = m2.class),
  @Relation(op = Op.DIV, lhs = m.class, rhs = s.class, res = mPERs.class)
\ttrcb{})
@Documented
@Retention(RetentionPolicy.RUNTIME)
@Target(\ttlcb{}ElementType.TYPE_USE, ElementType.TYPE_PARAMETER\ttrcb{})
@SubtypeOf(Length.class)
public @interface m \ttlcb{}
    Prefix value() default Prefix.one;
\ttrcb{}
\end{alltt}

Units Checker will automatically define the commutatively equivalent, and
inverse relationships. For the first relationship of \code{m * m = m2}, the
Units Checker automatically defines \code{m2 / m = m}. For the second
relationship of \code{m / s = mPERs}, the Units Checker automatically defines
\code{m / mPERs = s}, \code{s * mPERs = m}, and \code{mPERs * s = m}.

Equivalent dimension relationships must be defined for each custom unit
relationship. For example, \code{Length * Length = Area}.

If you want to define a custom unit that can be used with SI prefixes, simply
include \code{Prefix value() default Prefix.one;} within the definition of the
annotation. You can then define alias units of your custom unit using the
\code{@UnitsMultiple} meta-annotation.

If you want to define a pair of \code{TimeDuration} and \code{TimeInstant} unit,
you must also include the \code{@TimeRelation} meta-annotation in the definition
of the \code{TimeDuration} unit. For example, for the unit of years, we define:
\code{@TimeRelation(duration = year.class, instant = CALyear.class)}

Lastly, you will also have to provide constants that convert dimensionless types
to types that use the new unit. See \code{UnitsTools}'s source code for examples
(you will need to suppress a checker warning in just those few code locations
defining the constants).

See demonstration \code{docs/examples/units-extension/} for an example extension
that defines the unit Hertz (hz) as per second, the alias unit Kilohertz as per
millisecond, the dimension Frequency, and demo code which utilizes these three
annotations.

\section{What the Units Checker checks\label{units-checks}}

The Units Checker ensures that calculations and comparisons are correct with
respect to the units and dimensions of the operands.

All types with a particular unit or dimension annotation are disjoint from all
unannotated types, from all types with a different unit or dimension annotation,
and from all types with the same unit annotation but a different prefix.

Subtyping between the units and the dimensions is taken into account, as is the
\code{@UnitsMultiple} meta-annotation.

Multiplying a \code{@Dimensionless} with a unit type results in the same unit
type.

The division of a unit type by the same unit type results in the
\code{@Dimensionless} type.

Multiplying or dividing different unit types, for which no unit relation is
known to the system, will result in \code{@UnknownUnits}.

For most units, addition and subtraction of the same unit results in that unit.
Addition and subtraction of different units will result in \code{@UnknownUnits}.

Addition and subtraction follow a slightly different set of rules between
\code{TimeDuration} and \code{TimeInstant} units. For a pair of duration and
instant units, such as \code{@h} and \code{@CALh}, the following rules apply:

\begin{alltt}
@h + @h = @h
@CALh + @h = @CALh // eg 5am + 5h = 10am
@h + @CALh = @CALh
@CALh + @CALh = @UnknownUnits  // 5am + 10am = ??

@h - @h = @h
@CALh - @h = @CALh // eg 10am - 5h = 5am
@CALh - @CALh = @h // eg 10am - 5am = 5h
@h - @CALh = @UnknownUnits  // 5h - 10am = ??
\end{alltt}

Other respective pairs follow the same general rules.

If a pair is not directly related to each other, such as \code{@year} and
\code{@CALminute} then any addition or subtraction between them will result in
\code{@UnknownUnits}.

Lastly, comparisons are allowed only between types with the same unit, or
between any type and a number literal or null literal.

If you encounter a \code{@UnknownUnits} annotation in an error message, ensure
that your operations are performed on correct units or refine your
\code{UnitsRelations}.

The Units Checker does \textbf{\emph{not}} change units based on multiplication
with dimensionless numbers; for example, if variable \code{mass} has the type
\code{@kg double}, then \code{mass * 1000} has that same type rather than the
type \code{@g double}. (The Units Checker has no way of knowing whether you
intended a conversion, or you were computing the mass of 1000 items. You need to
make all conversions explicit in your code, and it's good style to minimize the
number of conversions.)

\section{Running the Units Checker\label{units-running}}

The Units Checker can be invoked by running the following commands.

\begin{itemize}

\item If your code uses only the SI units that are provided by the framework,
simply invoke the checker:

\begin{Verbatim}
  javac -processor org.checkerframework.checker.units.UnitsChecker MyFile.java ...
\end{Verbatim}

\item If you define your own units, provide the fully-qualified class names of
the annotations through the \code{-Aunits} option, using a comma-no-space-
separated notation:

\begin{alltt}
  javac -classpath \textit{/full/path/to/myProject/bin}:\textit{/full/path/to/myLibrary/bin} \ttbs
        -processor org.checkerframework.checker.units.UnitsChecker \ttbs
        -Aunits=\textit{myPackage.qual.MyUnit},\textit{myPackage.qual.MyOtherUnit} MyFile.java ...
\end{alltt}

The annotations listed in \code{-Aunits} must be accessible to the compiler
during compilation in the classpath. In other words, they must already be
compiled (and, typically, be on the javac classpath) before you run the Units
Checker with \code{javac}. It is not sufficient to supply their source files on
the command line.

\item You can also provide the fully-qualified paths to a set of directories
that contain units qualifiers through the \code{-AunitsDirs} option, using a
colon-no-space-separated notation. For example:

\begin{alltt}
  javac -classpath \textit{/full/path/to/myProject/bin}:\textit{/full/path/to/myLibrary/bin} \ttbs
        -processor org.checkerframework.checker.units.UnitsChecker \ttbs
        -AunitsDirs=\textit{/full/path/to/myProject/bin}:\textit{/full/path/to/myLibrary/bin} MyFile.java ...
\end{alltt}

Note that in these two examples, the compiled class file of the
\<myPackage.qual.MyUnit> and \<myPackage.qual.MyOtherUnit> annotations must
exist in either the \<myProject/bin> directory or the \<myLibrary/bin>
directory. The following placement of the class files will work with the above
commands:

\begin{alltt}
  .../myProject/bin/myPackage/qual/MyUnit.class
  .../myProject/bin/myPackage/qual/MyOtherUnit.class
\end{alltt}

The two options can be used at the same time to provide groups of annotations
from directories, and individually named annotations.

\end{itemize}

Also, see the invocation commands in the makefile of the example project in the
\<docs/examples/units-extension> directory.

\section{Suppressing warnings\label{units-suppressing}}

One example of when you need to suppress warnings is when you initialize a
variable with a unit type by a literal value. To remove this warning message, it
is best to introduce a constant that represents the unit and to add a
\code{@SuppressWarnings} annotation to that constant. For examples, see class
\code{UnitsTools}.

\section{References\label{units-references}}

\begin{itemize}
\item The GNU Units tool provides a comprehensive list of units:\\
  \url{http://www.gnu.org/software/units/}

\item The F\# units of measurement system inspired some of our syntax:\\
  \url{https://en.wikibooks.org/wiki/F_Sharp_Programming/Units_of_Measure}

\end{itemize}
